%% Requires compilation with XeLaTeX or LuaLaTeX
\documentclass[9pt,xcolor={table,dvipsnames},t,aspectratio=169,onlytextwidth,mathserif]{beamer}
\usetheme{PSU}
\setbeamercovered{still covered={\opaqueness<1->{0}},again covered={\opaqueness<1->{40}}}

\title[Your Short Title]{История Орловского государственного института им.Тургенева }
\subtitle{Орловского государственного университета имени И. С. Тургенева}
\author{Батова Алина}
\date{28.05.2024}

\begin{document}

\begin{frame}
  \titlepage
\end{frame}

% Uncomment these lines for an automatically generated outline.
% \begin{frame}{Outline}
%  % \tableofcontents
%      \begin{columns}[t]
%         \begin{column}{.45\paperwidth}
%             \tableofcontents[sections={1-2}]
%         \end{column}
%         \begin{column}{.45\paperwidth}
%             \tableofcontents[sections={3-}]
%         \end{column}
%     \end{columns}
% \end{frame}

\section{Introduction}

\begin{frame}{Истоки идеи о создании в Орле высшего учебного заведения впервые относятся к эпохе Первой мировой войны, когда, интеллигенция губернского центра, развивая инициативу Министерства промышленности и торговли 1916 г. в январе-марте 1917 г. создала «комиссию по устройству народного университета – политехникума». В мае того же года вопрос о его создании обсуждается Орловской городской комиссией по народному образованию. Итогом работы стал доклад, переданный 7 сентября 1918 г. Губернскому исполнительному комитету, содержавший вывод о необходимости создания в Орле университета с набором классических факультетов.}
\end{frame}


\section{Introduction}

\begin{frame}{ Институт открыл двери для 121 студента четырёх факультетов }

\centering
\begin{tabular}{l l}
\tableheadrow
\tableheadcol{Название} & \tableheadcol{Количество учащихся} \\
Физико-технический & 34 \\
Химико-биологический & 34\\ 
Литературно-общественный  & 23 \\
Политехнический & 30
\end{tabular}

\end{frame}

\section{Introduction}

\begin{frame}
{С 1934 г. (1 сентября) – в составе вуза структурно созданы два института – учительский с двухлетним сроком обучения и педагогический, с четырехлетним сроком обучения, выпускавший учителей для средних школ. В период сохранения данной структуры вплоть до 1952 г. вуз называется Орловский Государственный педагогический и учительский институт, сохраняя в прессе и документах внутреннего оборота название Орловский педагогический институт. При институте действует Рабфак (до 1949 г.), созданный в рамках ОГУ в 1920 г. и образцовая школа.}
\end{frame}

\section{Introduction}

\begin{frame}
\begin{figure}[h!]
\centering 
\includegraphics[width=0.6\linewidth]
{3ergfhUU7Ls.jpg}
\end{figure}
\end{frame}

\section{Introduction}

\begin{frame}
{25 ноября 2010 г. ОрёлГТУ переименован в Государственный университет — учебно-научно-производственный комплекс, а в 2015 г. в Приокский государственный университет.\\
В том же году принимается решение Правительства РФ о создании в Орловской области Опорного университета, во исполнение которого в 2016 г. (1 апреля) произошло создание на базе Университетов Опорного вуза «Орловский государственный университет имени И.С. Тургенева»}
\end{frame}
